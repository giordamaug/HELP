%% HEAT MAP macros
\usepackage{pgfplots}
\pgfplotsset{width=7cm,compat=1.8}
\usepackage{colortbl}
\usepackage{pgfplotstable}
\usepackage{listofitems}
\usepackage{ifthen}

\newcommand{\heatinv}{True}
\ifthenelse{\equal{\heatinv}{True}}{
\pgfplotstableset{
    bold/.append style={
        postproc cell content/.append code={
                \pgfkeysalso{@cell content=\textbf{##1}}%
        },
    },
    /color cells/min/.initial=0.0,
    /color cells/max/.initial=1.0,
    /color cells/mid/.initial=0.5,
    /color cells/col/.initial=,
    /color cells/table/.initial=,
    /color cells/textcolor/.initial=,
    %
    % Usage: 'color cells={min=, 
    %   max = }
    color cells/.code={%
        \pgfqkeys{/color cells}{#1}%
        \pgfkeysalso{%
            postproc cell content/.code={%
                %
                \begingroup
                % acquire the value before any number printer changed
                % it:
                \pgfkeysgetvalue{/pgfplots/table/@preprocessed cell content}\valueorig
                \setsepchar[.]{±}
                \readlist\myval{\valueorig}
                \renewcommand\value{\myval[1]}%
                \ifx\value\empty
                    \endgroup
                \else
                \pgfmathfloatparse{1.0-\value}
                %\pgfmathfloatparsenumber{1-\value}%
                \pgfmathfloattofixed{\pgfmathresult}%
                \let\ovalue=\pgfmathresult
                \pgfkeysgetvalue{/color cells/mid}\mmidval
                \pgfmathfloatparse{1.0-\mmidval}
                \pgfmathfloattofixed{\pgfmathresult}%
                \let\midval=\pgfmathresult
                \pgfmathparse{\ovalue>\pgfmathresult?int(0):int(1)}
                \ifnum\pgfmathresult>0\relax
                    \pgfkeyssetvalue{/color cells/textcolor}{white}
                \else
                \fi
                \pgfmathfloatparse{1.0-\pgfkeysvalueof{/color cells/max}}
                \let\minvalue=\pgfmathresult
                \pgfmathfloatparse{1.0-\pgfkeysvalueof{/color cells/min}}
                \let\maxvalue=\pgfmathresult
                % map that value:
                \pgfplotscolormapaccess
                    [\minvalue:\maxvalue]%[0.25:0.3]
                    {\ovalue}
                    {\pgfkeysvalueof{/pgfplots/colormap name}}%
                % now, \pgfmathresult contains {,,}
                % 
                % acquire the value AFTER any preprocessor or
                % typesetter (like number printer) worked on it:
                \pgfkeysgetvalue{/pgfplots/table/@cell content}\typesetvalue
                \pgfkeysgetvalue{/color cells/textcolor}\textcolorvalue
                %
                % tex-expansion control
                % see http://tex.stackexchange.com/questions/12668/
                \toks0=\expandafter{\typesetvalue}%
                \xdef\temp{%
                    \noexpand\pgfkeysalso{%
                        @cell content={%
                            \noexpand\cellcolor[rgb]{\pgfmathresult}%
                            \noexpand\definecolor{mapped
                              color}{rgb}{\pgfmathresult}%
                            \ifx\textcolorvalue\empty
                            \else
                                \noexpand\color{\textcolorvalue}%
                            \fi
                            \the\toks0 %
                        }%
                    }%
                }%
                \endgroup
                \temp
                \fi
            }%
        }%
    },
}
}{
\pgfplotstableset{
    bold/.append style={
        postproc cell content/.append code={
                \pgfkeysalso{@cell content=\textbf{##1}}%
        },
    },
    /color cells/min/.initial=0.0,
    /color cells/max/.initial=1.0,
    /color cells/mid/.initial=0.5,
    /color cells/col/.initial=,
    /color cells/table/.initial=,
    /color cells/textcolor/.initial=,
    %
    % Usage: 'color cells={min=, 
    %   max = }
    color cells/.code={%
        \pgfqkeys{/color cells}{#1}%
        \pgfkeysalso{%
            postproc cell content/.code={%
                %
                \begingroup
                % acquire the value before any number printer changed
                % it:
                \pgfkeysgetvalue{/pgfplots/table/@preprocessed cell content}\valueorig
                \setsepchar[.]{±}
                \readlist\myval{\valueorig}
                \renewcommand\value{\myval[1]}%
                \ifx\value\empty
                    \endgroup
                \else
                \pgfmathfloatparsenumber{\value}%
                \pgfmathfloattofixed{\pgfmathresult}%
                \let\value=\pgfmathresult
                \pgfkeysgetvalue{/color cells/mid}\midval
                \pgfmathparse{\value<\midval?int(1):int(0)}
                \ifnum\pgfmathresult>0\relax
                    \pgfkeyssetvalue{/color cells/textcolor}{white}
                \else
                \fi
                % map that value:
                \pgfplotscolormapaccess
                    [\pgfkeysvalueof{/color cells/min}:\pgfkeysvalueof{/color
                      cells/max}]
                    {\value}
                    {\pgfkeysvalueof{/pgfplots/colormap name}}%
                % now, \pgfmathresult contains {,,}
                % 
                % acquire the value AFTER any preprocessor or
                % typesetter (like number printer) worked on it:
                \pgfkeysgetvalue{/pgfplots/table/@cell content}\typesetvalue
                \pgfkeysgetvalue{/color cells/textcolor}\textcolorvalue
                % tex-expansion control
                % see http://tex.stackexchange.com/questions/12668/
                \toks0=\expandafter{\typesetvalue}%
                \xdef\temp{%
                    \noexpand\pgfkeysalso{%
                        @cell content={%
                            \noexpand\cellcolor[rgb]{\pgfmathresult}%
                            \noexpand\definecolor{mapped
                              color}{rgb}{\pgfmathresult}%
                            \ifx\textcolorvalue\empty
                            \else
                                \noexpand\color{\textcolorvalue}%
                            \fi
                            \the\toks0 %
                        }%
                    }%
                }%
                \endgroup
                \temp
                \fi
            }%
        }%
    },
}
}
\newcommand{\findmaxMCC}[1]{
    \setsepchar[.]{±}
    \pgfplotstableread[col sep = comma]{#1}\mytable
    \pgfplotstablesort[sort key={MCC},sort cmp={string >}]{\sorted}{\mytable}%
    \pgfplotstablegetelem{0}{MCC}\of{\sorted}%
    \readlist\MCCmax{\pgfplotsretval}
}
\newcommand{\findmaxACC}[1]{
    \setsepchar[.]{±}
    \pgfplotstableread[col sep = comma]{#1}\mytable
    \pgfplotstablesort[sort key={Accuracy},sort cmp={string >}]{\sorted}{\mytable}%
    \pgfplotstablegetelem{0}{Accuracy}\of{\sorted}%
    \readlist\ACCmax{\pgfplotsretval}
}
\newcommand{\findmaxROC}[1]{
    \setsepchar[.]{±}
    \pgfplotstableread[col sep = comma]{#1}\mytable
    \pgfplotstablesort[sort key={ROC-AUC},sort cmp={string >}]{\sorted}{\mytable}%
    \pgfplotstablegetelem{0}{ROC-AUC}\of{\sorted}%
    \readlist\ROCmax{\pgfplotsretval}
}
\newcommand{\findmaxBA}[1]{
    \setsepchar[.]{±}
    \pgfplotstableread[col sep = comma]{#1}\mytable
    \pgfplotstablesort[sort key={BA},sort cmp={string >}]{\sorted}{\mytable}%
    \pgfplotstablegetelem{0}{BA}\of{\sorted}%
    \readlist\BAmax{\pgfplotsretval}
}
\newcommand{\findmaxSN}[1]{
    \setsepchar[.]{±}
    \pgfplotstableread[col sep = comma]{#1}\mytable
    \pgfplotstablesort[sort key={Sensitivity},sort cmp={string >}]{\sorted}{\mytable}%
    \pgfplotstablegetelem{0}{Sensitivity}\of{\sorted}%
    \readlist\SNmax{\pgfplotsretval}
}
\newcommand{\findmaxSP}[1]{
    \setsepchar[.]{±}
    \pgfplotstableread[col sep = comma]{#1}\mytable
    \pgfplotstablesort[sort key={Specificity},sort cmp={string >}]{\sorted}{\mytable}%
    \pgfplotstablegetelem{0}{Specificity}\of{\sorted}%
    \readlist\SPmax{\pgfplotsretval}
}

\newcommand{\findminmaxMCC}[1]{
    \setsepchar[.]{±}
    \pgfplotstableread[col sep = comma]{#1}\mytable
    \pgfplotstablesort[sort key={MCC},sort cmp={string >}]{\sorted}{\mytable}%
    \pgfplotstablegetelem{0}{MCC}\of{\sorted}%
    \readlist\MCCmax{\pgfplotsretval}
    \pgfplotstablesort[sort key={MCC},sort cmp={string <}]{\sorted}{\mytable}%
    \pgfplotstablegetelem{0}{MCC}\of{\sorted}%
    \readlist\MCCmin{\pgfplotsretval}
}
\newcommand{\findminmaxACC}[1]{
    \setsepchar[.]{±}
    \pgfplotstableread[col sep = comma]{#1}\mytable
    \pgfplotstablesort[sort key={Accuracy},sort cmp={string >}]{\sorted}{\mytable}%
    \pgfplotstablegetelem{0}{Accuracy}\of{\sorted}%
    \readlist\ACCmax{\pgfplotsretval}
    \pgfplotstablesort[sort key={Accuracy},sort cmp={string <}]{\sorted}{\mytable}%
    \pgfplotstablegetelem{0}{Accuracy}\of{\sorted}%
    \readlist\ACCmin{\pgfplotsretval}
}
\newcommand{\findminmaxROC}[1]{
    \setsepchar[.]{±}
    \pgfplotstableread[col sep = comma]{#1}\mytable
    \pgfplotstablesort[sort key={ROC-AUC},sort cmp={string >}]{\sorted}{\mytable}%
    \pgfplotstablegetelem{0}{ROC-AUC}\of{\sorted}%
    \readlist\ROCmax{\pgfplotsretval}
    \pgfplotstablesort[sort key={ROC-AUC},sort cmp={string <}]{\sorted}{\mytable}%
    \pgfplotstablegetelem{0}{ROC-AUC}\of{\sorted}%
    \readlist\ROCmin{\pgfplotsretval}
}
\newcommand{\findminmaxBA}[1]{
    \setsepchar[.]{±}
    \pgfplotstableread[col sep = comma]{#1}\mytable
    \pgfplotstablesort[sort key={BA},sort cmp={string >}]{\sorted}{\mytable}%
    \pgfplotstablegetelem{0}{BA}\of{\sorted}%
    \readlist\BAmax{\pgfplotsretval}
    \pgfplotstablesort[sort key={BA},sort cmp={string <}]{\sorted}{\mytable}%
    \pgfplotstablegetelem{0}{BA}\of{\sorted}%
    \readlist\BAmin{\pgfplotsretval}
}
\newcommand{\findminmaxSN}[1]{
    \setsepchar[.]{±}
    \pgfplotstableread[col sep = comma]{#1}\mytable
    \pgfplotstablesort[sort key={Sensitivity},sort cmp={string >}]{\sorted}{\mytable}%
    \pgfplotstablegetelem{0}{Sensitivity}\of{\sorted}%
    \readlist\SNmax{\pgfplotsretval}
    \pgfplotstablesort[sort key={Sensitivity},sort cmp={string <}]{\sorted}{\mytable}%
    \pgfplotstablegetelem{0}{Sensitivity}\of{\sorted}%
    \readlist\SNmin{\pgfplotsretval}
}
\newcommand{\findminmaxSP}[1]{
    \setsepchar[.]{±}
    \pgfplotstableread[col sep = comma]{#1}\mytable
    \pgfplotstablesort[sort key={Specificity},sort cmp={string >}]{\sorted}{\mytable}%
    \pgfplotstablegetelem{0}{Specificity}\of{\sorted}%
    \readlist\SPmax{\pgfplotsretval}
    \pgfplotstablesort[sort key={Specificity},sort cmp={string <}]{\sorted}{\mytable}%
    \pgfplotstablegetelem{0}{Specificity}\of{\sorted}%
    \readlist\SPmin{\pgfplotsretval}
}
\newcommand{\heattable}[2]{
\pgfplotstableread[col sep = comma]{#1}\mytable
\findminmaxMCC{#1}
\findminmaxACC{#1}
\findminmaxROC{#1}
\findminmaxSN{#1}
\findminmaxSP{#1}
\findminmaxBA{#1}
%\noindent The MCC Max \MCCmax[1] and Min \MCCmin[1].       \\
%The Acc Max \ACCmax[1] and Min \ACCmin[1].\\
%The ROC Max \ROCmax[1] and Min \ROCmin[1].\\
%The SN \SNmax[1] and Min \SNmin[1].\\
%The SP \SPmax[1] and Min \SPmin[1].\\
%The BA \BAmax[1] and Min \BAmin[1].\\
%\addtolength{\tabcolsep}{-2pt}
\pgfplotstabletypeset[
        font=\small,
        every head row/.style={before row=\hline,after row=\hline},
        every odd row/.style={after row=\hline},
        every even row/.style={after row=\hline},
        %every last row/.style={before row=\hline,after row=\hline},
        columns={feature, ROC-AUC, Accuracy, BA, Sensitivity, Specificity}, %, MCC},
        assign column name/.style={/pgfplots/table/column name={\textbf{##1}}},
        every first column/.style={string type, column type = {l}}, 
        columns/ROC-AUC/.style={string type, column type = {|c}, color cells={col=Accuracy, min=\ROCmin[1],max=\ROCmax[1],mid=(\ROCmin[1]+\ROCmax[1])/2}},
        columns/Accuracy/.style={string type, column type = {|c}, color cells={col=Accuracy, min=\ACCmin[1],max=\ACCmax[1],mid=(\ACCmin[1]+\ACCmax[1])/2}},
        columns/BA/.style={string type, column type = {|c}, color cells={col=BA, min=\BAmin[1],max=\BAmax[1],mid=(\BAmin[1]+\BAmax[1])/2}},
        columns/Sensitivity/.style={string type, column type = {|c}, color cells={col=BA,min=\SNmin[1],max=\SNmax[1],mid=(\SNmin[1]+\SNmax[1])/2}},
        columns/Specificity/.style={string type, column type = {|c}, color cells={col=BA,min=\SPmin[1],max=\SPmax[1],mid=(\SPmin[1]+\SPmax[1])/2}},
        %columns/MCC/.style={string type, color cells={min=\MCCmin[1],max=\MCCmax[1],mid=(\MCCmin[1]+\MCCmax[1])/2}},
        col sep=comma,
        /pgfplots/colormap/#2,
        ]{\mytable}
}

\newcommand{\mycolormap}{blackwhite}